\begin{center}
\textbf{ABSTRAK}

\vspace{0.5cm}

\textbf{DETEKSI DAN ANALISIS EMISI DEBU DAN SINKROTRON DI LINGKUNGAN GALAKSI RADIO MELALUI 
CITRA KONTINUM \textit{DEEP-FIELD} DALAM RENTANG SUBMILIMETER DARI PENGAMATAN ALMA}\\
\vspace{0.5cm}

Oleh\\
\textbf{Nama Mahasiswa\\
NIM: 081320\\
Program Studi Doktor Astronomi}\\
\end{center}

\vspace{1.0cm}
Galaksi radio merupakan salah satu jenis {\it galaksi aktif} yang memancarkan energi sangat kuat pada panjang gelombang radio, dengan daya dalam rentang $10^{34}$ -- $10^{39}$ W. Pada umumnya mereka merupakan galaksi elips yang sangat masif, yang masih mengandung materi antar bintang dingin dan rapat. Keberadaan debu di galaksi radio dideteksi dari emisi inframerah-jauh (\textit{far-infrared}, FIR). Hal ini diduga dapat berasal dari proses pemanasan oleh bintang muda masif ataupun dari \textit{active galactic nuclei} (AGN). Berbagai mekanisme telah diusulkan untuk menjelaskan asal usul emisi tersebut, namun kontribusi dari masing-masing mekanisme masih belum dapat dipastikan.

Radio-AGN kuat biasanya mengandung sedikit gas molekuler dengan massa rata-rata sekitar $10^8$ M$_\odot$ pada galaksi induknya. Gas tersebut biasanya dideteksi sebagai pasokan ke cakram akresi yang mengelilingi inti, berupa lubang hitam sangat masif (SMBH), yang menghasilkan berbagai aktivitas inti. Dari beberapa penelitian lain, galaksi-galaksi tipe-awal tidak menunjukkan laju pembentukan bintang yang cepat, sehingga memberikan dugaan bahwa baik bintang-bintang tua ataupun AGN dapat menjadi sumber dari emisi FIR tersebut. Selain itu terdapat indikasi bahwa kandungan gas di galaksi elips tidak terkait dengan populasi bintang dan lebih menganjurkan pasokan gas yang berasal dari luar galaksi. Mekanisme yang membawa gas molekular ke daerah pusat diduga adalah proses \textit{merger} galaksi. Terdapat tiga jenis \textit{merger} berdasarkan kandungan gas yang dibawa, yaitu \textit{merger} basah (\textit{merger} antara dua galaksi kaya gas), kering, dan campuran.  Aktivitas galaksi radio bertipe Fanaroff-Riley I (FR I) diduga dipicu oleh akresi kering (\textit{Bondi accretion}), sedangkan tipe FR II dipicu oleh banyaknya pasokan gas yang masuk ke daerah pusat melalui proses \textit{minor-merger} basah.

Galaksi radio yang biasanya bertepatan dengan galaksi elips di optik, umumnya berada di tengah kumpulan galaksi yang cukup rapat, sehingga mempelajari sifat-sifat materi antar bintang (ISM) dari galaksi-galaksi di sekitar galaksi radio ini menjadi sangat penting guna memahami proses pembentukan galaksi radio, evolusi, serta umpan baliknya terhadap lingkungan. Untuk itu, dalam disertasi ini, kami mengusulkan untuk melakukan  studi terperinci lingkungan galaksi radio yang diamati oleh {\it Atacama Large Millimeter/submillimeter Array} (ALMA), yang sejauh ini merupakan teleskop radio paling sensitif dalam rentang panjang gelombang submilimeter. Data yang diperoleh dari pengamatan kalibrator untuk setiap proyek sains ALMA, yang sebagian besar merupakan galaksi radio, akan dianalisis secara terperinci. Setelah melalui proses seleksi terhadap target yang tepat dan dengan menggabungkan sejumlah data yang terkumpul, citra  kontinum submilimeter dengan tingkat derau yang sangat rendah ({\it deep-field submm images}) dalam orde puluhan $\mu$Jy haruslah dapat diperoleh. Dengan data tersebut emisi termal dan/atau sinkroton di bagian pusat galaksi haruslah dapat ditentukan. Demikian pula pengaruh distribusi ISM terhadap daerah pusat AGN beserta lingkungannya akan dapat dipelajari. Analisis selanjutnya akan dapat membantu membedakan sumber pemanasanan debu, apakah berasal dari pembentukan bintang ataukah dari AGN.


\vspace{1.0cm}

\noindent Kata kunci: AGN, galaksi radio, kalibrator, kontinum, submilimeter



