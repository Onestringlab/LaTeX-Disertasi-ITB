
\chapter{Introduction} 
\pagenumbering{arabic}

\section{Background}
Radio galaxies are a type of active galaxies that emit enormous radio luminosities, in the range of $10^{41}$ to $10^{46}$ erg s$^{-1}$ or equivalent to $10^{34}-10^{39}$ watts. They are one of the types of radio-loud active galaxy together with radio-loud quasar and blazar \citep{miley2008}. However, in this work, we use the  term ``radio galaxy'' for all types of radio-loud active galaxy. These objects can be detected at large distances, making them valuable tools for observational cosmology and the study of galaxy evolution.

Radio galaxies are divided mainly in two groups, the Fanaroff and Riley type I (FR-I) and type II (FR-II). As explained by \cite{fanaroff1974}, the sources were classified using the ratio of the distance between the regions of highest brightness of the radio continuum on opposite sides of the central galaxy or quasar, to the total extent of the source measured from the lowest contour; those sources from which the ratio was less than 0.5 were placed in class I, and those from which the ratio was greater than 0.5 were placed in class II. There is also a third classification called FR-c galaxies, covering galaxies that have a very compact radio continuum emission. Their radio morphologies suggest that they are compact versions of the classical FR-II's, although why they are so small is not yet established. It is hypothesized that these are either young FR-II's or FR-II's trapped in a dense environment \citep[e.g.][]{fanti1990, odea1991, fanti1994}.

The origin of the radio emission was studied in the last decade and still the complete picture about the physical processes are under debate \citep[e.g.][]{sikora1997, merloni2007, moderski1998}. In general, there are three types of merger of galaxy based on gas richness, wet (i.e. a merger between gas-rich galaxies), dry, and mixed merger. Based on the size of the galaxy, there are minor merger, when one of the galaxies is significantly larger than the other(s), and major merger, when two spiral galaxies that are approximately the same size collide. It has been argued that FR II radio galaxies, are triggered by molecular gas fed towards the center \citep{buttiglione2010} while the FR I type are triggered through a dry accretion (Bondi accretion) in the Broad Line Region \citep[BLR, see][]{fromerth2001}. It has been suggested that ``wet'' minor mergers would be a mechanism to bring molecular gas and dust towards the center of the radio galaxies and trigger the radio activity \citep[see, for example,][]{lim2000, israel1998}. A key aspect is the ability to feed the accretion disk around the central Supermassive Black Hole \citep[SMBH, see][]{merloni2008}.
 
The location of radio galaxies are normally overlap with giant elliptical galaxies with visual luminosities of about 2.1$\times$10$^{10} h^{-2}$ L$_\odot$  \citep{kellerman1988}. Although elliptical galaxies in general are relatively deficient in cold or molecular gas, observations of CO(J = 1$\rightarrow$ 0) emission lines indicate they still contain some dense and cold Interstellar Medium (ISM) \citep[e.g.,][]{wiklind1986}. Dust is also detected in those galaxies from the far-infrared (FIR) radiation. The radiation is assumed to be thermal and to originate from dust heated by either young massive stars or by the active galactic nuclei (AGN). According to \cite{kennicutt1998}, early-type galaxies -- correspond to elliptical galaxies -- show no independent evidence of high star formation rates (SFRs), suggesting either the older stars or the AGN are responsible for much of the FIR emission. \cite{wiklind1995} show that in elliptical galaxies, the gas is unrelated to the stellar populations, and favor an external origin of the molecular gas. 

It is then important to study the environment of radio galaxies and the ISM properties of the interstellar medium of the neighbor galaxies \citep{miley2008} to be able to understand the formation, the evolution and the feedback of the radio galaxy to their environment.  Measuring the molecular gas content in all the satellites would be, by far, too time consuming. However, the detection of the dust, a very good proxy of the global ISM, in a deep submm image would give valuable information on the amount of ISM  together with the star formation and/or possible activity in the neighbor galaxies. Moreover, it is possible to use only the radio  continuum emission since the molecular gas and dust are tightly linked by the correlation between IR luminosity as a tracer of dust, and HCN luminosity as a tracer of molecular gas in e.g. \cite{omont1996} and \cite{gaosolomon2004}. Such a detailed study is possible thanks to the advent of the Atacama Large Millimeter/submillimeter Array (ALMA), the largest radio interferometer array in the world, located in Northern Chile. Therefore, in this dissertation work, we propose to undertake a detailed study of the radio galaxy environments by using large amount of calibrator data provided by ALMA, available since the past few years. Detailed of the work will be described hereafter. 

\section{Objectives}

The objectives of this doctoral work will be:

\begin{enumerate}
\item To build a very deep catalogue of the dust emission in and around radio galaxies. This objective can be achieved by studying several radio galaxies covering a large redshift range from 0 to 2 with a very deep sensitivity on the order of a few tens of $\mu$Jy by using ALMA's calibrator data and obtaining the deepest image so far in continuum for the ALMA observing bands available at the time of this dissertation work (most likely in bands 3, 4, 6, 7, 8 and possibly 9), in the range of frequency from 84 GHz to 720 GHz.  

\item To uncover the physical origin of the radio continuum emission as being either thermal or synchrotron by building the Spectral Emission Distribution (SED) using the different ALMA bands. Once the discrimination is done, the main physical properties (dust mass, dust temperature, radio jet spectral index, etc) can be derived.

\item By using the built catalogue, we want to analyze the environment effects on the ISM content in these objects together with the effect of the galaxy evolution through the cosmic time (redshift). In particular, the dust content will be studied in the dense galaxy clusters hosting  the radio galaxies.

\item To analyze the synchrotron emission (radio jets) with the high angular resolution, in particular its interaction with the dust in the host elliptical galaxies and also in the environment.
\end{enumerate}

An in-depth analysis for individual radio galaxies will be performed depending on the features found in the corresponding deep sub-millimeter images. In addition, a sub-product of the work is a complete catalogue of the several absorption lines probably detected towards the strong radio continuum emission of these quasars.  These absorption lines presumably originate from our own Galaxy. Therefore, these data may be used to retrieve a map of molecular cloud in our Galaxy.

\section{Scope}
Scope of this project includes:

\subsection*{Data}
ALMA has been in operation since the end of 2011 by the Cycle 0 (Early Science Programs), conducted using 15 -- 20 antennas available at this period. Subsequently, the Cycle 1 was performed starting at the end of 2012, using many more antennas (more than 30), and so on. Presently, ALMA has been conducting the Cycle 5 for scientific community with more than 40 antennas. Accordingly, from then on, large data are available in public domain, since ALMA applied a short term proprietary period of one year only. Hence, all science projects will be available in public domain within one year after being delivered to the principal investigator. Correspondingly, for each science project, observations of different calibrators were performed (mainly radio galaxies or solar system objects) to derive a correct amplitude, phase and the frequency response of the ALMA correlators.

This research is aimed to produce a very deep (sensitivity in the order of a few tens of $\mu$Jy) catalogue of the dust emission in and around radio galaxies. However, the scope of our investigation will be limited to data from Cycle 1, 2, 3, and partly 4. Cycle 0 will not be explored since at the time, only a few number of antennas were used to collect the data, offering low sensitivity images which will not be significantly change the result if added to the more recent cycles data. Data from Cycle 5 is not publicly available yet, therefore can not be utilized for our project. We will  only select calibrators which have observational data at least from band 3, 6, and 7, to build the SED.

\subsection*{Cooperation}
This doctoral research work is a part of cooperation between Department of Astronomy, Faculty of Mathematics and Natural Sciences -- ITB and the European Southern Observatory (ESO) Part of the Joint ALMA Observatory (JAO), Santiago, Chile. Therefore, it is expected that some parts of this work will be conducted in Chile. 


\section{Method}

We note that every ALMA science project always includes calibration observations of very bright, compact sources to determine the flux density scale, to specify the bandpass response, and finally to calibrate the amplitude and phase of the visibilities  of the science targets. Most of bright radio sources are radio galaxies. Therefore, survey of these galaxies can be performed by analysing the data provided from calibration observations.

To achieve this goal, since the data are very big, we need to develop a pipeline that include maximizing the automatization of the data reduction. This includes the downloading, (self-)calibration, concatenation, as well as image ``cleaning''/filtering.  The final images will be denoised by applying a wavelet filtering 
\citep[e.g.,][]{leon2000} to improve the S/N.

After defining several criteria for selecting the calibrator samples, we should be able to study numerous radio galaxies in the range of  redshift from 0 to 2 (note that $z = 2$ corresponds to the age of the universe of about $\sim 3.3$ Gyr, using the standard model of cosmology) with a very deep sensitivity on the order of a few tens of $\mu$Jy. The point sources, and eventually extended sources, for the continuum emission will be extracted and analyzed using specific source extractor software.  Given the small statistics of the sources for the individual fields we will use a stacking strategy to analyze the radial distribution of the radio emission around the radio galaxies, eventually by binning in redshift range to study a possible variation.

Detected source in the environment of the radio galaxies  will be complemented by incorporating observations from other wavelengths, especially from  UV (star formation), optical (stars), infrared (dust), and radio wavelengths (radio jet). These data are in a form of image or database which available in public repository. 

\section{Novelty and Originality}

Our analysis will produce a new catalogue of very deep submm continuum emission around radio galaxies in the ALMA bands. As ALMA open new window in (sub)mm frequency with high spatial resolution and highest sensitivity, this catalogue will be very important in the study of the ISM in  and around radio galaxies together with the synchrotron emission (AGN). The different bands will provide valuable information about the ISM (dust properties, distribution) and the AGN properties (spectral index, flux) and the possible interplay between the ISM and the AGN (jet/dust interaction). We note that the data reduction pipeline will be quite unique to reduce extensively the ALMA observations with only another international group presently known working on similar data.

% a bit different in calibration process.

\section{Output Target}

During this work, we will disseminate our on-going research in seminars and several international conferences. Possible cooperation with other groups is also considered. Eventually, the output of this doctoral research will be published in several articles in international journal.

%\begin{itemize}
%\item 2 international publications
%\item 2 international conferences
%\item 1 thesis manuscript
%\end{itemize} 

\cleardoublepage
